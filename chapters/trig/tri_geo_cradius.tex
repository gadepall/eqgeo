%%
%\subsection{Perpendicular Bisectors}
\renewcommand{\theequation}{\theenumi}
\begin{enumerate}[label=\thesection.\arabic*.,ref=\thesection.\theenumi]
\numberwithin{equation}{enumi}
\item In 
	\figref{fig:tri-isosc},	
\begin{figure}[!ht]
	\begin{center}
		\resizebox{\columnwidth}{!}{%Code by GVV Sharma
%July 6, 2023
%Revised July 7, 2023
%released under GNU GPL
%The Isosceles Triangle

\begin{tikzpicture}
[scale=2,>=stealth,point/.style={draw,circle,fill = black,inner sep=0.5pt},]

%Triangle sides
\def\a{5}
\def\b{6}
\def\c{4}
 
%Coordinates of A
%\def\p{{\a^2+\c^2-\b^2}/{(2*\a)}}
\def\p{0.5}
\def\q{{sqrt(\c^2-\p^2)}}

%Labeling points
%\node (A) at (\p,\q)[point,label=above right:$A$] {};
\node (B) at (0, 0)[point,label=below left:$B$] {};
\node (C) at (\a, 0)[point,label=below right:$C$] {};

%Circumcentre

\node (O) at (2.5,1.70084013)[point,label=above right:$O$] {};

%Drawing triangle OBC
%\draw (A) -- node[left] {$\textrm{c}$} (B) -- node[below] {$\textrm{a}$} (C) -- node[above,yshift=2mm] {$\textrm{b}$} (A);
%Drawing OA, OB, OC
%\draw (O) -- node[left] {$\textrm{R}$} (A);
\draw (O) -- node[below] {${R}$} (B);
\draw (O) -- node[below] {${R}$} (C);
\draw (B) -- node[below] {${a}$} (C);

\tkzMarkAngle[fill=blue!50,size=.3](C,B,O)
\tkzMarkAngle[fill=blue!50,size=.3](O,C,B)


\tkzMarkAngle[fill=red!10](B,O,C)
\tkzLabelAngle[pos=0.3](B,O,C){$\theta$}
%\tkzMarkAngle[fill=red!10](A,C,O)

\iffalse
\tkzMarkAngle[fill=orange!50,size=.3](B,A,O)
\tkzMarkAngle[fill=orange!50,size=.3](O,B,A)

\tkzLabelAngle[pos=0.5](O,C,B){$\theta_1$}
\tkzLabelAngle[pos=0.5](O,B,C){$\theta_1$}
\tkzLabelAngle[pos=0.5](O,A,B){$\theta_2$}
\tkzLabelAngle[pos=0.5](O,B,A){$\theta_2$}
\tkzLabelAngle[pos=1.5](O,A,C){$\theta_3$}
\tkzLabelAngle[pos=1.5](O,C,A){$\theta_3$}
\fi

\end{tikzpicture}
}
	\end{center}
	\caption{Isosceles Triangle}
	\label{fig:tri-isosc}	
\end{figure}
\begin{align}
	OB = OC=R
\end{align}
Such a triangle is known as an isosceles triangle.  Show that
\begin{align}
	\angle B = \angle C
\end{align}
\solution 
Using
\eqref{eq:tri_sin_form},
\begin{align}
	\frac{\sin B}{R} &= \frac{\sin C}{R}
	\\
\implies	{\sin B} &= {\sin C}
\\
	\text{or, } \angle B &= \angle C.
\end{align}
\item In 
	\figref{fig:tri-isosc},	
	show that 
  \begin{align}
	  a = 2R \sin\frac{ \theta }{2}
\label{eq:crad_cos2a}
  \end{align}
		\solution In $\triangle OBC$,  using the cosine formula from
\eqref{eq:tri_cos_form},
\begin{align}
	\cos \theta &= \frac{R^2+R^2 - a^2}{2R^2} = 1 -\frac{a^2}{2R^2}
	\\
	\implies \frac{a^2}{2R^2}&= 2\sin^2\frac{\theta}{2}
\end{align}
yielding 
\eqref{eq:crad_cos2a}.
\item In
	\figref{fig:tri_ccircle-ang},
show that 
\begin{align}
\label{eq:tri_crad_R}
\frac{a}{\sin A} = \frac{b}{\sin B} = \frac{c}{\sin C} = 2R.
\end{align}
%
%
\solution
From 
\eqref{eq:ang-subtend-ccentre}
and 
\eqref{eq:crad_cos2a}
  \begin{align}
	  a = 2R \sin A
  \end{align}

\item In 
	\label{prob:tri-ccentre-def}
	\figref{fig:tri-perp-bis}, 
\begin{align}
OB = OC=R, 	BD = DC.
\end{align}
Show that $OD \perp BC$.
%
\begin{figure}[!ht]
	\begin{center}
		
		\resizebox{\columnwidth}{!}{%Code by GVV Sharma
%July 7, 2023
%released under GNU GPL
%The perpendicular bisector

\begin{tikzpicture}
[scale=2,>=stealth,point/.style={draw,circle,fill = black,inner sep=0.5pt},]

%Triangle sides
\def\a{5}
\def\b{6}
\def\c{4}
 
%Coordinates of A
%\def\p{{\a^2+\c^2-\b^2}/{(2*\a)}}
\def\p{0.5}
\def\q{{sqrt(\c^2-\p^2)}}

%Labeling points
%\node (A) at (\p,\q)[point,label=above right:$A$] {};
\node (B) at (0, 0)[point,label=below left:$B$] {};
\node (C) at (\a, 0)[point,label=below right:$C$] {};
%Mid point
\node (D) at ($(B)!0.5!(C)$)[point,label=below:$D$] {};

%Circumcentre

\node (O) at (2.5,1.70084013)[point,label=above right:$O$] {};

%Drawing triangle OBC
%\draw (A) -- node[left] {$\textrm{c}$} (B) -- node[below] {$\textrm{a}$} (C) -- node[above,yshift=2mm] {$\textrm{b}$} (A);
%Drawing OA, OB, OC
%\draw (O) -- node[left] {$\textrm{R}$} (A);
\draw (O) -- node[below] {${R}$} (B);
\draw (O) -- node[below] {${R}$} (C);
\draw (B) -- (C);
%\draw (B) -- node[below] {${a}$} (C);
\draw (O) --   (D);

\tkzMarkAngle[fill=blue!50,size=.3](C,B,O)
\tkzMarkAngle[fill=blue!50,size=.3](O,C,B)


%\tkzMarkAngle[fill=red!10](O,A,C)
%\tkzMarkAngle[fill=red!10](A,C,O)

\iffalse
\tkzMarkAngle[fill=orange!50,size=.3](B,A,O)
\tkzMarkAngle[fill=orange!50,size=.3](O,B,A)

\tkzLabelAngle[pos=0.5](O,C,B){$\theta_1$}
\tkzLabelAngle[pos=0.5](O,B,C){$\theta_1$}
\tkzLabelAngle[pos=0.5](O,A,B){$\theta_2$}
\tkzLabelAngle[pos=0.5](O,B,A){$\theta_2$}
\tkzLabelAngle[pos=1.5](O,A,C){$\theta_3$}
\tkzLabelAngle[pos=1.5](O,C,A){$\theta_3$}
\fi

\end{tikzpicture}
}
	\end{center}
	\caption{Perpendicular bisector.}
	\label{fig:tri-perp-bis}	
%github/geometry/figs/
\end{figure}
\\
\solution 
\begin{align}
	\norm{\vec{O}-\vec{C}} &=
\norm{\vec{O}-\vec{B}} =R
\\
	\implies \norm{\vec{O}-\vec{C}}^2 &=
\norm{\vec{O}-\vec{B}}^2 
\end{align}
which can be expressed as 
\begin{align}
%  \label{eq:norm2d_dist}
	\brak{\vec{O}-\vec{C}}^{\top} \brak{\vec{O}-\vec{C}}&=
	\brak{\vec{O}-\vec{B}}^{\top} 
\brak{\vec{O}-\vec{B}}
\\
\norm{\vec{O}}^2-2{\vec{O}}^{\top}\vec{C} + \norm{\vec{C}}^2
	&= \norm{\vec{O}}^2-2{\vec{O}}^{\top}\vec{B} + \norm{\vec{B}}^2
	\\
	\implies 
	  \brak{\vec{B}-\vec{C}}^{\top}{\vec{O}} 
	  &=  \frac{\norm{\vec{B}}^2 - \norm{\vec{C}}^2}{2}
\end{align}
which can be simplified to obtain
  \begin{align}
	  \brak{\vec{B}-\vec{C}}^{\top}\cbrak{{\vec{O}}- 
	    \brak{\frac{{\vec{B}} + {\vec{C}}}{2}}}=0
  \label{eq:norm2d_equidist}
  \\
	  \text{or, }
	  \brak{\vec{B}-\vec{C}}^{\top}\cbrak{{\vec{O}}- \vec{D}}=0
  \label{eq:norm2d_equidist-conv}
  \end{align}
  which proves the give result using 
	  \eqref{eq:section_formula}
	  and 
\eqref{eq:tri_cos_form-orth}.
\item In 
	\figref{fig:tri_ccentre},
$OD$ and $OE$ are the perpendicular bisectors of sides $BC$ and $AC$ respectively.  Show that 
$OA = R$.
\begin{figure}[!ht]
	\begin{center}
		
		\resizebox{\columnwidth}{!}{%Code by GVV Sharma
%December 9, 2019
%released under GNU GPL
%Locating the circumcentre

\begin{tikzpicture}
[scale=2,>=stealth,point/.style={draw,circle,fill = black,inner sep=0.5pt},]

%Triangle sides
\def\a{5}
\def\b{6}
\def\c{4}
 
%Coordinates of A
%\def\p{{\a^2+\c^2-\b^2}/{(2*\a)}}
\def\p{0.5}
\def\q{{sqrt(\c^2-\p^2)}}

%Labeling points
\node (A) at (\p,\q)[point,label=above right:$A$] {};
\node (B) at (0, 0)[point,label=below left:$B$] {};
\node (C) at (\a, 0)[point,label=below right:$C$] {};

\node (D) at ($(B)!0.5!(C)$)[point,label=below:$D$] {};
\node (E) at ($(A)!0.5!(C)$)[point,label=right:$E$] {};
%Circumcentre

\node (O) at (2.5,1.70084013)[point,label=left:$O$] {};

%Drawing triangle ABC
\draw (A) -- node[left] {} (B) -- node[below] {} (C) -- node[above,yshift=2mm] {} (A);
%Drawing OA, OB, OC
\draw (O) -- node[left] {} (A);
\draw (O) -- node[below] {$\textrm{R}$} (B);
\draw (O) -- node[below] {$\textrm{R}$} (C);
\draw (O) --   (D);
\draw (O) --   (E);

\tkzMarkRightAngle[fill=blue!30,size=.2](C,D,O)
\tkzMarkRightAngle[fill=blue!30,size=.2](O,E,C)
%\tkzMarkAngle[fill=blue!50,size=.3](O,C,B)


%\tkzMarkAngle[fill=red!10](O,A,C)
%\tkzMarkAngle[fill=red!10](A,C,O)


%\tkzMarkAngle[fill=orange!50,size=.3](B,A,O)
%\tkzMarkAngle[fill=orange!50,size=.3](O,B,A)
%
%\tkzLabelAngle[pos=0.5](O,C,B){$\theta_1$}
%\tkzLabelAngle[pos=0.5](O,B,C){$\theta_1$}
%\tkzLabelAngle[pos=0.5](O,A,B){$\theta_2$}
%\tkzLabelAngle[pos=0.5](O,B,A){$\theta_2$}
%\tkzLabelAngle[pos=1.5](O,A,C){$\theta_3$}
%\tkzLabelAngle[pos=1.5](O,C,A){$\theta_3$}

\end{tikzpicture}
}
	\end{center}
	\caption{ Perpendicular bisectors of $\triangle ABC$ meet at $\vec{O}$.}
	\label{fig:tri_ccentre}	
\end{figure}
\\
\solution Tracing
  \eqref{eq:norm2d_equidist-conv}
  backwards yields
\begin{align}
	OB = OC, OC = OA = R.
\end{align}
\end{enumerate}
